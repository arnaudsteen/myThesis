\chapter{Simulation du SDHCAL}
\label{chap.simulation}
La simulation est un aspect très important dans les expériences de physique des particules. En effet, la conception et l'optimisation d'un détecteur s'appuient toujours sur la simulation qui va permettre une estimation rapide des performances, des coûts de l'expérience. Les simulations sont aussi massivement utilisées dans l'analyse des données pour améliorer les algorithmes d'analyses, pour confirmer ou non la présences de nouvelle physique, pour estimer les biais... C'est pourquoi avoir une simulation la plus réaliste possible est un enjeu très important. Dans ce chapitre, nous présenterons les modèles utilisés pour la simulation des gerbes hadroniques. Puis nous détaillerons la simulation du prototype SDHCAL et nous expliquerons les différentes étapes de la modélisation de la réponse des GRPC aux particules chargées. Enfin quelques des comparaisons entre les données et la simulation seront présentées. 
\minitoc
\newpage

%%%%%%%%%%%%%%%%%%%%%%%%%%%%%%%%%%%%%%%%%%%%%%%

\section{Les modèles de simulation des gerbes hadroniques}
La collaboration GEANT4~\cite{geant4} fournit un logiciel rassemblant de nombreux modèles théoriques et phénoménologiques qui décrivent les interactions des particules avec la matière. Ces modèles n'étant valable que sur certaines gammes d'énergies (cf. figure~\ref{fig.model}), l'utilisateur doit les combiner pour simuler un phénomène physique. La collaboration GEANT4 fournit aussi un certain nombre de listes physiques définissant et utilisant des transitions entre ces modèles (cf. section~\ref{sec.listphys}).
\begin{figure}[!ht]
  \begin{center}
    \includegraphics[width=.8\textwidth]{Digitizer/figs/HadronicModelsInventory.jpg}
    \caption{Les modèles utilisés dans GEANT4 en fonction de l'énergie des particules.}
    \label{fig.model}
  \end{center}
\end{figure}
\\
Les principaux modèles pour les interactions hadroniques de haute énergie avec la matière sont les modèles de cordes partoniques (cf. section~\ref{sec.parton}) qui sont valables pour des énergies supérieures à $5-10$ GeV. Les interactions aux énergies intermédiaires (100 MeV < E < 10 GeV) sont simulées avec les modèles de cascades intranucléaires (cf. section~\ref{sec.inucl}). Pour traiter les noyaux excités par des collisions de plus haute énergie et les interactions en dessous de 200 MeV, une famille de modèles de désexcitation nucléaire (fission, évaporation nucléaire\dots) est disponible. Les interactions des neutrons de basses énergies (E<20 MeV) peuvent être simulées avec des modèles de haute précision pour neutrons où un grand nombre de sections efficaces ont été tabulées. L'utilisation ou non de ces modèles de haute précision pour les neutrons aura des conséquences sur le temps de calcul, sur la réponse simulée du détecteur ou sur la topologie des gerbes.
\subsection{Modèles de cordes partoniques}
\label{sec.parton}
Les modèles de cordes partoniques permettent de simuler les réactions de hautes énergies de hadrons avec des noyaux. Les deux principaux modèles utilisés dans GEANT4 sont les modèles QGS (Quark Gluon String) et FTF (Fritiof) \cite{geant4_parton}. Le résultat de l'interaction d'un hadron avec un noyau est une ou plusieurs cordes excitées. Une corde est un segment où chacune des deux extrémités est un quark ou un di-quark se déplaçant dans des directions opposées. Les noyaux sont modélisés comme un ensemble de nucléons dont les positions sont aléatoirement choisies en utilisant une distribution de densité. Pour les noyaux lourd~(A>16), une distribution de densité de la forme du potentiel de Wood-Saxon est utilisée : 
\begin{equation}
  \label{eq.wood-saxon}
  \rho(r_i)=\frac{\rho_0}{1+exp[(r_i-R)]/a}
\end{equation}
où $R$ et $a$ dépendent de la masse du noyau. Pour les noyaux légers une distribution de densité venant du modèle d'oscillateur harmonique est utilisée : 
\begin{equation}
  \label{eq.harmonic-ocsillator}
  \rho(r_i)=(\pi R'^2)^{-3/2}exp(-r_i^2/R'^2)
\end{equation}
avec $R'$  qui dépend de la masse du noyau.
Pour calculer le paramètre d'impact avec les nucléons, ces deux distributions de densité sont réduites dans un plan perpendiculaire à la direction de la particule incidente. La probabilité de collision entre entre le hadron et un nucléon est calculée en utilisant une distribution gaussienne pour les fonctions d'onde du hadron et des nucléons. Ces probabilités sont utilisée pour connaître le nombre de nucléons participant à la réaction dans le noyau.\\
Des cordes sont ensuite crées, les quarks du hadron incident sont aléatoirement répartis entre celles-ci. Un modèle de fragmentation longitudinal de cordes est ensuite utilisé où ces cordes sont séparées en hadrons et en une nouvelle corde. Une corde se fragmente en une paire quark-antiquark $q-\bar q$ ou diquark-antidiquark $qq-\bar q \bar q$ \cite{geant4_reference}. Les probabilités relatives de création des quark ou diquark sont :
\begin{equation}
  u:d:s:qq = 1:1:0.35:0.1
\end{equation}
La paire de quark-antiquark (ou diquark-antidiquark) créée est placée entre la précédente paire. Une moitié de ces nouvelles paires sont utilisées pour créer un hadron tandis que les autres constituants créent une nouvelle corde. Ce processus se répète jusqu'à ce que l'énergie d'une corde ne soit pas suffisante pour créer un hadron. Le tableau~\ref{tab.partonModelTable} présente les domaines de validité des modèles QGS et FTF pour différentes particules incidentes.
\begin{table}[!ht]
  \begin{center}
    \begin{tabular}{c|c|c}
      Particule & QGS & FTF\\
      \hline
      $k+$ & $12\ GeV\ -\ 100\ TeV$ & $4\ GeV\ -\ 100\ TeV$\\
      $k-$ & $12\ GeV\ -\ 100\ TeV$ & $4\ GeV\ -\ 100\ TeV$\\
      $\lambda$ & $ $ & $2\ GeV\ -\ 100\ TeV$\\
      $\pi+$ & $12\ GeV\ -\ 100\ TeV$ & $4\ GeV\ -\ 100\ TeV$\\
      $\pi-$ & $12\ GeV\ -\ 100\ TeV$ & $4\ GeV\ -\ 100\ TeV$\\
      $neutron$ & $12\ GeV\ -\ 100\ TeV$ & $4\ GeV\ -\ 100\ TeV$\\
      $proton$ & $12\ GeV\ -\ 100\ TeV$ & $4\ GeV\ -\ 100\ TeV$\\
      $ion$ & $ $ & $2\ GeV\ -\ 100\ TeV$\\
    \end{tabular}
  \end{center}
  \caption{Domaine de validité des modèles QGS et FTF pour plusieurs hadrons incidents.}
  \label{tab.partonModelTable}
\end{table}

Après l'interaction du noyau avec la particule incidente, celui-ci sera dans un état excité. Le retour à l'état fondamental du noyau est simulée avec des modèle de fragmentation et de désexcitation nucléaire.
\subsection{Le modèles de cascade intranucléaire de Bertini}
\label{sec.inucl}
Il a été montré dans \cite{bertini} en 1969 qu'un modèle de cascade intranucléaire décrivait relativement bien les interactions de nucléons de 100 MeV à 2 GeV avec des noyaux. Ces réactions sont caractérisées par une rapide ($10^{-23}-10^{-22} s$) cascade intranucléaire laissant les noyaux dans un état excité, suivi d'une phase plus lente ($10^{-18}-10^{-16} s$) d'évaporation nucléaire. Le modèles de cascade binaire et celui de Bertini sont les modèles disponibles dans GEANT4 pour simuler les cascades intranucléaire.
\subsubsection{Bertini}
\begin{figure}[!ht]
  \begin{center}
    \includegraphics[width=.8\textwidth]{Digitizer/figs/intraNucl.png}
    \caption{Schéma d'explication du modèle de Bertini. Un hadron de 400 MeV crée une cascade intranucléaire.}
    \label{fig.g4bertini}
  \end{center}
\end{figure}
.Le modèle de Bertini a été étendu et est valable pour des particules incidentes ($p$, $n$, $\pi$, $K$, $\Delta$, $\Sigma$, $\Xi$, $\Omega$ et $\gamma$) avec une énergie cinétique comprise entre 100 MeV et 10 GeV~\cite{geant4_bertini}. Ce modèle est applicable lorsque la longueur d'onde de de Broglie de la particule incidente est de l'ordre de la distance moyenne entre les nucléons du noyau. Le noyau cible est modélisé par une ou trois couches concentriques de densité constante en fonction du nombre de nucléons dans le noyau (1 couche si A<4, 3 sinon). La cascade commence lorsque une particule incidente rencontre un nucléon du noyau cible. Le point d'impact de la particule incidente est choisit aléatoirement dans une distribution sphérique uniforme. Les sections efficaces entre la particule et les nucléons, la densité de nucléons et les impulsions des nucléons sont utilisés pour calculer le libre parcours de la particule. Les impulsions des nucléons sont calculées en utilisant le modèle du gaz de Fermi. Une collision entre la particule incidente et un nucléon peut produire des particules secondaires. Pour les pions, le modèle de Bertini prend en charge les collisions élastiques, les collisions inélastiques suivantes: $\pi^-p\rightarrow\pi^0n$, $\pi^0p\rightarrow\pi^+n$, $\pi^0n\rightarrow\pi^-p$, et $\pi^+n\rightarrow\pi^0p$. Des réactions produisant plus de 2 particules sont aussi implémentées. Les pions peuvent aussi être absorbés par les nucléons via des réactions de la forme: $\pi^-pp\rightarrow np$. L'impulsion du nucléon et des particules secondaires créées sont calculées. Ces particules secondaires sont alors susceptibles d'interagir à leur tour avec les nucléons du noyau si leur énergie cinétique est supérieur à 2 $MeV$. La valeur de cette coupure vient du principe d'exclusion de Pauli. Les produits d'une réaction initié par une particule d'énergie inférieure à 2 $MeV$ auraient une énergie encore plus faible. Or dans un gaz de Fermi, les niveaux d'énergie sont remplis à partir du plus bas niveau. L'énergie minimal pour la production de nouvelles particules correspond au plus faible niveau d'énergie non rempli. Pour simplifier le modèle et tenir compte de principe de Pauli, la coupure à 2 $MeV$ et utilisée. La cascade intranucléaire prend fin lorsque toutes les particules secondaires sont absorbées ou se sont échappées du noyau. Le noyau est alors dans un état excité: des nucléons du noyau ont changé de niveau d'énergie. Un modèle exciton est alors utilisé pour traité les transitions de ces nucléons. Les noyaux résultants peuvent être instables et seront traités avec des modèles de fission et d'évaporation nucléaire.

\subsubsection{Le modèle de cascade binaire}
Dans GEANT4, un autre modèle de cascade intranucléaire est disponible et est utilisé par certaine liste physique. C'est le modèle de cascade binaire. Dans ce modèle, les nucléons du noyau cible sont au repos. La densité des noyaux est de la forme du potentiel de Wood-Saxon (equation~\ref{eq.wood-saxon}) pour les noyaux lourd (A>16). Pour les noyaux légers la densité donnée par le modèle d'oscillateur harmonique est utilisée (equation~\ref{eq.harmonic-ocsillator}). La trajectoire de la particule primaire et des secondaires sont des lignes droites. À chaque étape de la cascade, la plus proches distances $d_i^{min}$ entre le nucléon $i$ et la trajectoire des particules est calculée pour tous les nucléons du noyau. Des collisions entre les particules et les noyaux sont possibles si cette distance satisfait la condition suivante: $d_i^{min}<\sqrt{\frac{\sigma_i}{\pi}}$, où $\sigma_i$ est la section efficace d'interaction entre le nucléon cible et la particule. De même que pour le modèle de Bertini, le principe d'exclusion de Pauli est vérifié et si une interaction crée une particule avec une énergie inférieure à un seuil, cette réaction est supprimée. La cascade prend fin lorsque toutes les particules secondaires s'échappent du noyau ou l'énergie des particules secondaires est insuffisante pour continuer la cascade.

\subsubsection{Les modèles paramétrés}
Dans GEANT4, deux modèles étaient utilisés pour simuler l'ensemble des processus de manière paramétrée. Ce sont les modèles Low Energy Parametrized (LEP) et High Energy Parametrized (HEP). Ces modèles on montré des limites. Les résolutions en énergie des gerbes hadroniques simulées avec ces modèles étaient souvent meilleurs que dans les données expérimentales. Les simulations produisaient également des gerbes hadroniques plus étroites que les données. Le modèle LEP est encore utilisé dans quelques listes physiques (cf. section\ref{sec.listPhys}) mais est progressivement remplacé.
\subsection{Les listes physiques}
\label{sec.listPhys}
Les différents modèles implémentés dans GEANT4 ne sont valables que sur certaines gammes d'énergie. La simulation d'un phénomène tel que les gerbes hadroniques a besoin de modèles valides sur une très grande gamme d'énergie. La particule primaire peut pénétrer le calorimètre avec une énergie allant de quelque $GeV$ jusqu'à plus de 100 $GeV$. Les modèles de GEANT4 sont donc combiner entre eux dans des listes physiques. Une liste physique est définit par plusieurs modèles et par les lois de transitions entre ces modèles. La figure~\ref{fig.g4list} présente les deux liste physiques les plus utilisées dans le domaine de la calorimétrie à haute énergie. 
\label{sec.listphys}
\begin{figure}[!ht]
  \begin{center}
    \includegraphics[width=.8\textwidth]{Digitizer/figs/physics_list_G4.pdf}
    \caption{Schéma descriptif des listes physiques FTFP\_BERT(\_HP) et QGSP\_BERT(\_HP).}
    \label{fig.g4list}
  \end{center}
\end{figure}
Les transitions entre les modèles sont linéaires. Pour la liste FTFP\_BERT, GEANT4 va aléatoirement choisir entre les modèles de Bertni et Fritiof pour simuler l’interaction d'une particule d'énergie entre 4 et 5 $GeV$. Par exemple, le modèle Bertini aura une probabilité de 25$\%$ d'être choisis alors que la probabilité de choisir le modèle de Fritiof sera de 75$\%$. Il est a noter que la gamme de validité d'un modèle peu varier dans une liste selon la nature de la particule. Enfin, précisons que pour quelques listes physiques, l'option $HP$ (High Precision for neutrons) est disponible. L'utilisation de cette option permet d'être plus précis sur les interactions de neutrons avec la matière lorsque ceux-ci ont une énergie inférieure à 20 $MeV$. Cette option augmente le temps de calcul de la simulation du SDHCAL, d'un facteur légèrement inférieur à 10. Le tableau suivant présente plusieurs listes physiques avec les domaines de validité des différents modèles pour les protons.
\begin{table}[!ht]
  \begin{center}
    \begin{tabular}{c|c|c}
      Liste physique & Modèle & Gamme d'énergie \\
      \hline
      FTFP\_BERT(\_HP) & Bertini & 0$\rightarrow$4 $GeV$\\
      $ $        & Fritiof & 5$\rightarrow$100000 $GeV$\\
      \hline
      QGSP\_BERT(\_HP) & Bertini & 0$\rightarrow$9.9 $GeV$\\
      $ $        & LEP & 9.5$\rightarrow$25 $GeV$\\
      $ $        & QGS & 12$\rightarrow$100000 $GeV$\\
      \hline
      QGSP\_FTFP\_BERT & Bertini & 0$\rightarrow$8 $GeV$\\
      $ $              & Fritiof & 6$\rightarrow$25 $GeV$\\
      $ $              & QGS & 12$\rightarrow$100000 $GeV$\\
      \hline
      QGSP\_BIC(\_HP) & Cascade Binaire & 0$\rightarrow$9.9 $GeV$\\
      $ $             & LEP & 9.5$\rightarrow$25 $GeV$\\
      $ $             & QGS & 12$\rightarrow$100000 $GeV$\\
      \hline
      FTF\_BIC  & Cascade Binaire & 0$\rightarrow$ 5 $GeV$\\
      $ $       & FTF & 4$\rightarrow$100000 $GeV$\\
    \end{tabular}
  \end{center}
  \caption{Exemple de listes physiques disponibles dans GEANT4. L'option $HP$ est indiquée lorsqu'elle est disponible. Cette option n'est utilisée que pour les neutrons d'énergie inférieure à 20 $MeV$.}
  \label{tab.physList}
\end{table}

%%%%%%%%%%%%%%%%%%%%%%%%%%%%%%%%%%%%%%%%%%%%%%%

\section{La simulation du prototype}
La simulation du prototype a été réalisé avec un programme basé sur GEANT4. La géométrie du détecteur est décrite dans ce programme avec de nombreux détails. Les compositions chimiques et les densités des matériaux du prototypes sont détaillés. Ces informations seront utilisées par GEANT4 pour des calculs de sections efficaces lors de la propagation des particules dans le détecteur. Le champ électrique entre les deux électrodes de verre des GRPC n'est pas simulé. Les avalanches issues des ionisations du gaz par les particules incidentes sont modélisés dans un algorithme dédié qui sera décrit dans la partie~\ref{sec.digit-algo} de ce chapitre. Cet algorithme est aussi responsable de la répartition de la charge sur les carreaux de cuivre et donc de la multiplicité (voir section~\ref{sec.muons} du chapitre~\ref{chap.sdhcal}). 
La trajectoire des particules primaires et secondaires est segmentée dans GEANT4. A chaque interaction avec les matériaux du détecteur, un nouveau segment est créé. La création d'un nouveau segment se fait aussi à chaque fois qu'une particule change de volume (e.g. passant du verre au gaz). Il arrive que plusieurs segments soient créés pour une seule particule dans une couche de gaz. Ces segments pourraient déclencher la simulation de plusieurs avalanches dans une couche de gaz et ainsi modifier la réponse simulée du détecteur. Les segments appartenant à la même particule dans une même couche de gaz sont reliés entre eux. 
La liste des segments après correction, se trouvant dans le milieu actif du détecteur (le gaz entre les électrodes pour le SDHCAL), est alors enregistrée. Seulement les segments associés à des particules chargées sont conservés. Les informations stockées pour ces segments sont les suivantes: les coordonnées des positions du début et de la fin du segment; l'énergie perdue par la particule le long du segment; la nature de la particule; le temps d’occurrence du segment relatif au moment où la particule primaire a été générée. 
Cette liste de segment est ensuite utilisée comme point de départ de l’algorithme qui modélise la réponse des GRPC au particules chargées.

%%%%%%%%%%%%%%%%%%%%%%%%%%%%%%%%%%%%%%%%%%%%%%%

\section{Modèlisation de la réponse des GRPC au particules chargées}
Nous venons de voir que le résultat du programme de simulation du SDHCAL est une liste de segments correspondant à une partie de la trajectoire d'une particule dans le détecteur. Nous avons vu que l'énergie déposée par ces segments étaient disponible alors que dans le cas du SDHCAL la charge induite sur les carreaux de cuivre est mesurée. De plus, le phénomène de multiplicité introduit au chapitre~\ref{chap.sdhcal} n'est pas pris en compte par GEANT4. 
\label{sec.digit-algo}
%%%%%%%%%%%%%%%%%%%%%%%%%%%%%%%%%%%%%

\subsection{Algorithme SimDigital}
\label{sec.algo}
La modélisation est faite à l'aide d'un algorithme appelé SimDigital qui est un processeur Marlin \cite{marlin} disponible dans le paquet MarlinReco \cite{marlinreco} de l'ILCSoft \cite{ilcsoft}. Le but de cet algorithme est de simuler la réponse des GRPC lors du passage de particules chargées dans la l'intervalle de gaz. Les différentes étapes de l’algorithme sont les suivantes:
\begin{enumerate}[~~1-]
\item La fenêtre en temps utilisée dans la procédure de reconstruction des événements décrites dans la partie~\ref{sec.trivent} du chapitre~\ref{chap.sdhcal} est de 1000 $ns$. Ainsi, les particules interagissant tardivement dans le détecteur comme les neutrons peuvent ne pas être associées à l’événement. Pour prendre cet effet en compte, les segments avec un temps d’occurrence supérieur à 1000 $ns$ sont supprimés.
\item \label{it.start} Un canal de lecture $C_0$ où un ou plusieurs segments ont traversé le gaz est sélectionné. La longueur de ses segments est alors calculée.
\item La longueur de certains des segments dans le gaz peut être très petite. Ce phénomène peut arriver aléatoirement lors de la propagation des particules par GEANT4. Cependant la majorité des cas où ce phénomène s'observe, s'explique par le changement de volume d'une particule. La figure~\ref{fig.map_and_length_vs_deltaz}(a) présente la longueur des segments en fonction de la distance ($\Delta_z$) entre la position du milieu du segment et le milieu de la couche de gaz. Cette figure montre qu'une grande fraction des segments de faible longueur sont proches des deux électrodes en verre ($|\Delta_z|\simeq0.6~mm$). Ces segments de longueur presque nulle n'ont pas de raison de déclencher une avalanche dans le gaz. Les segments de longueur inférieure à une longueur donnée $l_{min}$ sont donc supprimés.
  \begin{figure}[!ht]
    \subfigure[]{\includegraphics[width=0.5\textwidth]{Digitizer/figs/stepLength_zoom.pdf}}
    \subfigure[]{\includegraphics[width=0.5\textwidth]{Digitizer/figs/layer19map.pdf}}
    \caption{(a): Longueur des segments (step length) en $mm$ en fonction de $\Delta_z$ en $mm$. Cette figure est centrée sur la région des segments de faible longueur pour mettre en évidence le fait que la plupart des segments de longueur nulle (ou presque) sont localisés sur les bords de la couche de gaz ($|\Delta_z|\simeq0.6\ mm$). (b): Exemple d'une carte d'efficacité des ASICs.}
    \label{fig.map_and_length_vs_deltaz}
  \end{figure}
\item Les cartes d'efficacité des ASIC du prototype déterminées en utilisant la même méthode que celle décrite dans la partie~\ref{sec.muons} du chapitre~\ref{chap.sdhcal}, sont utilisées pour prendre en compte les effets des $quenchers$ ($CO_2$,$SF_6$). En effet, les propriétés de ces deux gaz (capture des photons et d'électrons) ne sont pas incluses dans GEANT4. De plus, l'utilisation de ces cartes d’efficacité permet d'éviter la présence de signal dans les canaux électroniques hors d'usage ou masqués. Ainsi, lorsqu'un segment est dans une région du détecteur où l'efficacité est de 50\% alors ce segment a 50\% de chance d'être conservé. La figure~\ref{fig.map_and_length_vs_deltaz} montre un exemple de carte d'efficacité d'une chambre du prototype SDHCAL.
\item La charge induite pour chaque segment est alors aléatoirement choisis dans une distribution de Polya: 
  \begin{equation}
    \label{eq.polya}
    P(q)=[\frac{q}{\bar q}(1+\theta)]^{\theta}e^{[-\frac{q}{\bar q}(1+\theta)]}
  \end{equation}
  où $\bar q$ est la valeur moyenne et $\theta$ un paramètre libre lié à la largeur de la distribution. Cette charge induite est ensuite corrigée: 
  \begin{equation}
    \label{eq.lengthcorrection}
    Q_{corr} = \left\{ \begin{array}{rl}
      Q_{ind}(\frac{d_s}{d_{gap}})^\kappa &\mbox{ si $\frac{d_s}{d_{gap}}>1$} \\
      Q_{ind} &\mbox{ sinon}
    \end{array} \right.
  \end{equation}
  où $d_s$ correspond à la longueur du segment, $d_{gap}$ à l'épaisseur de la couche de gaz (1.2 $mm$) et $\kappa$ un paramètre libre. L'effet et l'importance de cette correction seront discuté dans la partie~\ref{sec.param} de ce chapitre.
\item Dans les gerbes hadroniques et électromagnétiques, les particules chargés peuvent être très proches. Ainsi les avalanche induites dans le gaz par ces particules peuvent se chevaucher. Cependant, dans le régime avalanche le signal induit par ces particules n'est pas équivalent à la somme des signaux induits par ces particules pris individuellement. Pour simuler cet effet, lorsque la distance entre deux segments est inférieure à une valeur $d_{cut}$, le segment dont la charge induite ($Q_{corr}$) est la plus faible, est supprimé. %Ce filtre permet aussi de supprimer un ou plusieurs segments provenant d'une seule particule. En effet, une interaction entre une particule chargée et le gaz peut avoir lieu dans GEANT4. Ainsi plusieurs segments seraient créés et pourraient déclencher plusieurs avalanches alors qu'une seule particule est passée.
\item \label{it.spliting} La charge induite est ensuite réparti sur le canal de lecture $C_0$  et sur les canaux voisins. Les canaux voisins correspondent au cellules dans le même plan que $C_0$ à une distance inférieur à une valeur $r_{max}$. Le rapport suivant est alors calculé pour chaque canal:
  \begin{equation}
    \label{eq.ratio}
    R_i = \frac{\int_{a_i}^{b_i}\int_{c_i}^{d_i}\sum_{j=0}^{n}\alpha_j e^{ \frac{(x_0-x)^2+(y_0-y)^2}{\sigma_j^2}}dxdy}{N}
  \end{equation}
  où $a_i,\ b_i,\ c_i,\ d_i$ sont les positions des bords de la cellule $i$, $(x_0,y_0)$ sont les coordonnées du milieu du segment et N un facteur de normalisation définit comme: 
  \begin{equation}
    \label{eq.norm}
    N=\int_{-r_{max}}^{+r_{max}}\int_{-r_{max}}^{+r_{max}}\sum_{j=0}^{n}\alpha_j e^{ \frac{(x_0-x)^2+(y_0-y)^2}{\sigma_j^2}}dxdy
  \end{equation}
  Pour chaque canal ($C_0$ et ses voisins), sa charge induite est augmentée d'une valeur $R_iQ_{corr}$.
\item L'opération est répétée à partir de l'étape~\ref{it.start} pour tous les canaux avec des segments.
\item Les seuils sont finalement appliqués pour tous les canaux. Les canaux pour lesquels la charge induite est inférieur à la valeur du premier seuil sont supprimés. Les canaux restant sont alors étiquetés selon la valeur de leur charge induite. 
\end{enumerate}

%%%%%%%%%%%%%%%%%%%%%%%%%%%%%%%%%%%%%

\subsection{Paramétrisation de l'algorithme}
\label{sec.param}
Nous avons vu les différentes étapes de l'algorithme responsable de la simulation de la réponse des GRPC au passage de particules chargées. Cet algorithme introduit de nombreux paramètres. Les méthodes utilisés pour obtenir la meilleur paramétrisation sont décrites dans la section suivantes. 
%%%%%%%%%%%%%%%%%%%%%%%%%%%%%%%%%%%%%

\subsubsection{Mesure du spectre de charge}
% 
\label{sec.polya}
Le régime utilisé pour les GRPCs du prototype est le régime avalanche saturée. Ce régime à été décrit dans \cite{abbresciaPolya} et il a été montré qu'une distribution de Polya (cf. Eq.~\ref{eq.polya}) s'ajuste bien aux spectres de charge expérimentales. Il n'était pas possible de faire une mesure directe du spectre de charge avec une chambre du prototype. Cette mesure qui nécessite une lecture analogique d'une GRPC avait été réalisé avec une chambre différentes de celles du prototype. La chambre utilisé était plus petite, le signal était enregistré avec un seul canal de lecture de $8 \times 8 ~cm^{2}$. Le mélange de gaz et la haute tension appliquée était différents. Nous étions donc contraints de refaire la mesure du spectre de charge pour les GRPCs du prototype avec une autre méthode. La méthode que nous avons utilisé est un scan en seuil. Cette méthode consiste à étudier l'efficacité de détection de muons en fonctions de la valeur des seuils. Nous avons choisis 9 chambres dans lesquels nous avons fait varier la valeur du seuil 1, 2 ou 3.
\begin{table}[!ht]
  \begin{center}
    \begin{tabular}{c|c}
      Seuil & Numéro de chambre\\
      \hline
      $1$ & $6,~16,~30$\\
      $2$ & $10,~22,~34$\\
      $3$ & $14,~26,~38$\\
    \end{tabular}
  \end{center}
  \caption{Liste des chambres utilisées pour le scan en seuil.}
  \label{tab.thrScan}
\end{table}
Le tableau~\ref{tab.thrScan} montre quel seuil nous avons fait varier dans quelle chambre. Pour étudier l'efficacité de détection de ces chambres, nous avons utilisé les autres pour reconstruire les traces des muons. La méthode de reconstruction des muons décrite dans le chapitre~\ref{chap.sdhcal} est de nouveau utilisée. La figure~\ref{fig.thrScan}(a) montre les efficacités moyennes en fonction du seuil. Cette courbe est ensuite ajustée avec le fonction suivante:
\begin{equation}
  \label{eq.fitScan}
  \varepsilon(q)=\varepsilon _0 - c\int_0^q{[\frac{q'}{\bar q}(1+\theta)]^{\theta}e^{[-\frac{q'}{\bar q}(1+\theta)]}dq'}
\end{equation}
où $\bar q$ et $\theta$ sont les paramètre de la distribution de Polya, $c$ est un paramètre libre et $\varepsilon_0$ la valeur asymptotique de l'efficacité.
\begin{figure}[!ht]
  \subfigure[]{\includegraphics[width=.5\textwidth]{Digitizer/figs/thrScanDat.pdf}}
  \subfigure[]{\includegraphics[width=.5\textwidth]{Digitizer/figs/thrScanSim.pdf}}
  \caption{(a) Résultats du scan en seuil pour les données (a) et pour la simulation (b). \label{fig.thrScan}}
\end{figure}
 La même procédure est réalisé avec la simulation. Les paramètre $\bar q$ et $\theta$ sont réglés pour reproduire les résultats obtenus avec les données du prototype. La figure~\ref{tab.thrScan}(b) montre la courbe d'efficacité moyenne en fonction du seuil pour la simulation. Le tableau~\ref{tab.inputPolya} présente le valeur d'entrée des paramètres de la distribution de Polya utilisée dans l'algorithme.
\begin{table}[!ht]
  \begin{center}
    \begin{tabular}{c|c}
      Paramètre & Valeur\\
      \hline
      $\bar q$ & $1.12$\\
      $\theta$ & $4.58 pC$\\
    \end{tabular}
  \end{center}
  \caption{Valeur d'entrée des paramètres de la Polya utilisé par l'algorithme SimDigital.}
  \label{tab.inputPolya}
\end{table}
 Cependant, les valeurs des paramètres obtenues avec l'ajustement (cf. Eq.~\ref{eq.fitScan}) sont sensiblement différentes des valeurs d'entrées. Ceci vient de l'étalement de la charge induite sur plusieurs canaux de lecture (cf. étape~\ref{it.spliting} de la section~\ref{sec.algo} de ce chapitre). Ainsi une fraction de la charge induite est perdue lorsque les seuils sont appliqués.
%%%%%%%%%%%%%%%%%%%%%%%%%%%%%%%%%%%%%

\subsubsection{Répartition de la charge}
Les paramètres introduits dans les équations~\ref{eq.ratio} et~\ref{eq.norm} sont réglés pour reproduire la multiplicité moyenne et le nombre de coups dans les gerbes électromagnétiques. La multiplicité moyenne pour chaque GRPC est calculée en utilisant la même méthode de reconstruction de trace décrit dans la chapitre~\ref{chap.sdhcal}. De nombreuses configurations de ces paramètres ont été testées pour reproduire à la fois la multiplicité moyenne et le nombres de coups dans les gerbes électromagnétiques. Le paramètre $n$ de l'équation~\ref{eq.ratio} est fixé à 2. Il n'était pas possible de reproduire les différentes observables des données avec $n=1$. Une augmentation de la valeur de ce paramètre n'apporte pas d'amélioration et ajoute des difficultés pour paramétrer l'algorithme. Le tableau~\ref{tab.splitting} présentes les valeurs d'entrées des paramètres $\alpha$ et $\sigma$.
\begin{table}[!ht]
  \begin{center}
    \begin{tabular}{c|c}
      Paramètre & Valeur\\
      \hline
      $\alpha_0$ & $1.0$\\
      $\sigma_0$ & $1.0~mm$\\
      $\alpha_1$ & $0.00083$\\
      $\sigma_1$ & $9.7~mm$\\
    \end{tabular}
  \end{center}
  \caption{Valeur d'entrée des paramètres introduit dans l'équation~\ref{eq.ratio}.}
  \label{tab.splitting}
\end{table}
Le paramètre $r_{max}$ est fixé à 30 $mm$. Une augmentation de cette valeur n’entraîne pas de variation significative sur le résultat final de la simulation. En effet, la quantité de charge déposée sur les carreaux éloignés de plus de 30 $mm$ de la particule est négligeable (avec la paramétrisation actuelle). La figure~\ref{fig.eff_mul_layer} montre l'efficacité (fig.~\ref{fig.eff_mul_layer}(a)) et la multiplicité (fig.~\ref{fig.eff_mul_layer}(b)) moyenne par chambre pour les données et la simulation.
\begin{figure}[!ht]
  \subfigure[]{\includegraphics[width=.5\textwidth]{Digitizer/figs/effLayer.pdf}}
  \subfigure[]{\includegraphics[width=.5\textwidth]{Digitizer/figs/mulLayer.pdf}}
  \caption{Efficacité (a) et multiplicité (b) par plan. Les données sont représentées par des cercles noirs et la simulation par des carrés rouges.\label{fig.eff_mul_layer}}
\end{figure}
L'efficacité dans la simulation suit raisonnablement bien les fluctuations des données car les cartes d'efficacités (déterminés avec les données) sont utilisés par l'algorithme. En revanche, les fluctuations de multiplicité ne sont pas reproduites par la simulation. Les différences de multiplicité d'une chambre à l'autre peuvent s'expliquer par des différences de résistivité de la peinture appliquée sur les verres et par des imperfections de la géométrie des détecteurs. De plus, lors des test en faisceaux de 2015 au PS au CERN, l'étude de scan en seuil a de nouveau été réalisé. Cette fois, nous avons fait varier les seuils dans toutes les chambres. Ainsi, une distribution de Polya peut-être extraite pour chaque détecteur. La figure~\ref{fig.polya_ps_2015} montre l'efficacité de détection de deux GRPCs du prototype en fonction du seuil. Ces efficacités sont aussi ajustées avec la fonction~\ref{eq.fitScan}.
\begin{figure}[!ht]
  \includegraphics[width=.5\textwidth]{Digitizer/figs/layer8.pdf}
  \includegraphics[width=.5\textwidth]{Digitizer/figs/layer37.pdf}
  \caption{Exemple d'efficacité en fonction du seuil pour deux chambres du prototype}
  \label{fig.polya_ps_2015}
\end{figure}
Les paramètres de la Polya, indiqués sur ces deux figures, sont très différents. L'utilisation d'une distribution de Polya différente pour chaque chambre devrait permettre à la multiplicité simulée de respecter les fluctuations des données expérimentales.
%%%%%%%%%%%%%%%%%%%%%%%%%%%%%%%%%%%%%

\subsubsection{Dépendance de l'angle d'incidence}
Lors des tests sur faisceau, les muons incidents sont pour la plupart perpendiculaires au détecteurs. Cependant, dans les gerbes hadroniques et électromagnétiques des particules secondaires sont émises avec des angles différents. Une étude de la multiplicité avec des particules cosmiques est alors nécessaire pour déterminer et simuler la réponse d'une GRPC sur une large gamme d'angles d'incidence. La figure~\ref{fig.mul_vs_theta}(a) montre la multiplicité en fonction de $cos{\theta}$ où $\theta$ est l'angle entre la normale au détecteur (axe $(0z)$ pour le prototype) et la particule incidente. Cette figure montre que la multiplicité pour les données expérimentales, augmente avec l'angle de la particule incidente alors qu'elle est plus plate pour la simulation. Une correction de la charge simulée est nécessaire pour reproduire le comportement des données. Une correction utilisant l'équation~\ref{eq.lengthcorrection} est préférée à une correction utilisant "$\frac{1}{cos\theta'}$" (avec $\theta'$ l'angle entre le segment et la normale au détecteur) car cette dernière peut générer des charges simulées infinies. 
\begin{figure}[!ht]
  \subfigure[]{\includegraphics[width=.5\textwidth]{Digitizer/figs/mul_vs_thetaNLC.pdf}}
  \subfigure[]{\includegraphics[width=.5\textwidth]{Digitizer/figs/mul_vs_theta.pdf}}
  \caption{Multiplicité moyenne en fonction de $cos\theta$ avec des cercles noirs et des carré rouges pour les données et la simulation respectivement. (a): sans correction sur la longueur des segments; (b) avec correction sur la longueur des segments. \label{fig.mul_vs_theta}}
\end{figure}
La figure~\ref{fig.mul_vs_theta}(b) montre un bon accord entre les données et la simulation pour la multiplicité en fonction de $cos\theta$ après l'application de cette correction. La valeur du facteur $\kappa$ est fixée à 0.40.
%% \begin{figure}[!ht]
%%   \begin{center}
%%     \includegraphics[width=.8\textwidth]{Digitizer/figs/costhetaPI30.pdf}
%%     \caption{Distribution du cos de l'angle de la step avec la normal à la GRPC pour une simulation d'un échantillon de pion à 30 GeV.}
%%     \label{fig.g4list}
%%   \end{center}
%% \end{figure}

%%%%%%%%%%%%%%%%%%%%%%%%%%%%%%%%%%%%%

\subsubsection{Paramétrisation des seuils}
Nous avons vu dans la section~\ref{sec.sdhcal_thr} du chapitre~\ref{chap.sdhcal} comment les seuils étaient réglés dans le prototype. Cependant comme ces travaux n'ont pas pu être effectuer avec des détecteurs complet, il est probable que les valeurs de conversion (entre DAC et valeurs de seuil en $pC$) soient légèrement différentes pour le prototype. Ceci donne un peu de liberté pour régler les seuils dans la simulation. La figure~\ref{fig.thrScan}(b) montre qu'une faible variation du premier seuil ($seuil\in[0.1,0.4]~pC$) a des conséquences négligeable sur l'efficacité ($\varepsilon\in[0.94,0.95]$). Ainsi, la valeur du premier seuil utilisé dans la simulation est calculé avec l'équation~\ref{eq.dacConversion}. Pour régler la valeur des seuils supérieur, nous avons de nouveau réalisé une étude d'efficacité. La même méthode décrite dans le chapitre~\ref{chap.sdhcal} est utilisé pour déterminer si un plan est efficace. Lorsqu'un plan est efficace, il est aussi considéré comme efficace pour les seuils 2 et/ou 3 si au moins un coup ayant passé ces seuils sont trouvés dans le groupe de coups correspondant. Les seuils 2 et 3 sont alors régler pour reproduire les efficacités des données expérimentales. 
\begin{figure}[!ht]
  \subfigure[]{\includegraphics[width=.5\textwidth]{Digitizer/figs/eff2Layer.pdf}}
  \subfigure[]{\includegraphics[width=.5\textwidth]{Digitizer/figs/eff3Layer.pdf}}
  \caption{Efficacité pour le deuxième (a) et le troisième (b) seuil par plan. Les données sont représentées par des cercles noirs et la simulation par des carrés rouges.\label{fig.eff_thr}}
\end{figure}
La figure~\ref{fig.eff_thr} montre les efficacités par plan pour les seuils 2~(fig.~\ref{fig.eff_thr}(a)) et 3~(fig.~\ref{fig.eff_thr}(b)) pour la simulation et pour les données. Les seuils 2 et 3 sont fixés à 5.4 et 14.5 $pC$ respectivement pour la simulation alors que pour les données l'équation~\ref{eq.dacConversion} indique des valeurs de seuils de 5.0 et 15.0 $pC$~($DAC_2=500$; $DAC_3=345$).
%%%%%%%%%%%%%%%%%%%%%%%%%%%%%%%%%%%%%

\subsubsection{Résumé}
La valeur de deux paramètres n'a pas été encore discuté. Le paramètre $l_{min}$ est fixé à 1 $\mu m$. Nous avons vérifié que l'effet de petite variation autour de cette valeur est négligeable. Le paramètre $d_{cut}$ est fixé à 0.5 $mm$. Il a été régler pour reproduire le nombre de coups dans les gerbes électromagnétiques (c.f. section~\ref{sec.resultats}). Le tableau~\ref{tab.summary} contient la liste des paramètres introduits dans l'algorithme et leur valeur.
\begin{table}[!ht]
  \begin{center}
    \begin{tabular}{c||c}
      Paramètre & Valeur \\
      \hline
      \hline
      $l_{min}$ & $0.001\ mm$\\
      $d_{cut}$ & $0.5\ mm$ \\
      \hline
      $\bar q$ & $4.58\ pC$ \\
      $\theta$ & $1.12$ \\ 
      \hline
      $n$ & $2$ \\ 
      $r_{max}$ & $30\ mm$ \\
      $\alpha_0$ & $1.0$ \\
      $\sigma_0$ & $1.0\ mm$ \\
      $\alpha_1$ & $0.00083$ \\
      $\sigma_1$ & $9.7\ mm$ \\
      \hline
      $\kappa$ & $0.40$\\
      \hline 
      $seuil_1$ & $0.114\ pC$\\
      $seuil_2$ & $5.4\ pC$\\
      $seuil_3$ & $14.5\ pC$
    \end{tabular}
  \end{center}  
  \caption{Paramètres d'entrée de l'algorithme SimDigital.}
  \label{tab.summary}
\end{table}
%%%%%%%%%%%%%%%%%%%%%%%%%%%%%%%%%%%%%

\subsection{Résultats}
\label{sec.resultats}
Nous avons vu les différentes méthodes utilisées pour régler les valeurs des paramètres introduits dans l'algorithme SimDigital. Nous allons maintenant tester cet algorithme avec des gerbes électromagnétiques puis hadroniques. Dans cette section, les mêmes coupures sont appliquées sur les échantillons de données pour filtrer les muons du faisceaux, les particules cosmiques et les particules neutres que dans la section~\ref{sec.shower_selection} du chapitre~\ref{chap.sdhcal}. Les coupures appliquées sur les échantillons de données pour rejeter les événements électrons sont aussi les mêmes que celles du chapitre~\ref{chap.sdhcal}. La procédure de sélection des gerbes électromagnétiques sera décrite dans la sous partie correspondante. Les coupures appliquées aux données expérimentales sont aussi appliquées sur les échantillons de simulation afin d'éviter des biais. Les listes physiques utilisées dans la suite sont FTFP\_BERT\_HP et QGSP\_BERT\_HP. Enfin, rappelons que le nombre de coups dans les gerbes est calibré avec le temps (voir section~\ref{sec.timeCalib} du chapitre~\ref{chap.sdhcal}).

%%%%%%%%%%%%%%%%%%%%%

\subsubsection{Gerbes électromagnétique}
 Les gerbes électromagnétiques sont traditionnellement bien simulées par GEANT4. Les comparaisons entre les données expérimentales et la simulation des gerbes électromagnétiques n'auront pas pour objectif de valider ou d'invalider les modèles de GEANT4 mais plutôt de vérifier la qualité de l'algorithme introduit pour modéliser la réponse des GRPCs au passage de particules chargées. De plus certains paramètres de l’algorithme ont été réglé pour reproduire la réponse du détecteur lors du passage d'une gerbe électromagnétique.
Pour décider si une gerbe a été induite par un électron, les trois critères suivant doivent être vérifié:
\begin{enumerate}[~~1-]
\item Le nombre de plan avec au moins 1 coup doit être inférieur à 30.% (fig.~\ref{fig.e-_begin_layer}(a)).
\item Le nombre de traces reconstruites avec la technique de Transformée de Hough (voir section~\ref{sec.hough} du chapitre~\ref{chap.topo}) doit être nulle. %\textcolor{red}{(!!!référence au chapitre sur les topo quand il sera écrit!!!)}
\item Le premier plan d’interaction doit être compris dans un des cinq premiers plans.% (fig.~\ref{fig.e-_begin_layer}(b)).
%  \begin{figure}[!ht]
%    \subfigure[]{\includegraphics[width=.5\textwidth]{Digitizer/figs/Nlayer_e-_50GeV.pdf}}
%    \subfigure[]{\includegraphics[width=.5\textwidth]{Digitizer/figs/Begin_e-_50GeV.pdf}}
%    \caption{(a): Distribution du nombre plans avec au moins un coup pour un échantillon de données de gerbes électromagnétiques à 50 GeV (la coupure sur le nombre de plan n'a pas été appliquée sur cette figure). (b): Distribution du premier plan d’interaction pour un échantillon de données de gerbes électromagnétiques à 50 GeV (la coupure sur le premier plan d'interaction n'a pas été appliquée sur cette figure)\label{fig.e-_begin_layer}}
%  \end{figure}
\end{enumerate}
La figure~\ref{fig:electron_control} dans le chapitre~\ref{chap.sdhcal} montre les distributions de chacune de ces variables pour un échantillon de simulation de gerbes électromagnétiques à 50 $GeV$. Le tableau~\ref{tab.e_sel_eff} présente l'efficacité de sélection des gerbes électromagnétiques en fonction de l'énergie. L'efficacité est mesurée avec la simulation et correspond au rapport entre le nombre d'événements simulés et le nombre d'événements identifiés comme gerbes électromagnétiques.
\begin{table}[!ht]
  \begin{center}
    \begin{tabular}{c|c}
      Energie & Efficacité \\
      \hline
      $10~GeV$ & $99.3\%$ \\
      $20~GeV$ & $99.4\%$ \\
      $30~GeV$ & $99.3\%$ \\
      $40~GeV$ & $99.2\%$ \\
      $50~GeV$ & $99.0\%$ \\
    \end{tabular}
  \end{center}  
  \caption{Efficacité de sélection des gerbes électromagnétique en fonction de l'énergie de la particule incidente.}
  \label{tab.e_sel_eff}
\end{table}
La figure~\ref{fig.e-Selection} montre les distributions de nombre de coups pour des échantillons de données d'électrons à 10 GeV (a) et 50 (b) GeV (GeV) avant les coupures, après les coupures de sélection des électrons et après celles de pions.
\begin{figure}[!ht]
  \subfigure[]{\includegraphics[width=.5\textwidth]{Digitizer/figs/selection715725.pdf}}
  \subfigure[]{\includegraphics[width=.5\textwidth]{Digitizer/figs/selection715716.pdf}}
  \caption{Distribution du nombre de coups pour des échantillons de données d'électrons à 10 GeV (a) et 50 GeV (GeV). Les lignes noires montrent les distributions de nombres de coups avant les coupures, les lignes rouges après les coupures de sélection des électrons et les lignes bleu après les coupures de sélection des pions. \label{fig.e-Selection}}
\end{figure}
Rappelons que certaines gerbes hadroniques étaient rejetées par les coupures pour filtrer les électrons (voir section~\ref{sec.pi_selection} du chapitre~\ref{chap.sdhcal}). Cet effet était surtout présents à basse énergie (E<20 $GeV$). Les échantillons de données de gerbes électromagnétiques sont probablement toujours contaminés par des gerbes hadroniques malgré l'application de ces coupures. La fraction électromagnétique des gerbes hadroniques identifiées comme électromagnétiques a été étudiée sur des échantillons de simulation. Elle est approximée en mesurant le rapport de l'énergie déposée par des électrons, des positrons ou des photons sur l'énergie totale déposée dans le calorimètre. La valeur moyenne de cette fraction électromagnétique pour les gerbes hadroniques identifiées gerbes électromagnétiques est proche de 80$\%$ à 10 $GeV$.
\begin{figure}[!ht]
    \subfigure[]{\includegraphics[width=.5\textwidth]{Digitizer/figs/nhit_e-_20GeV_AugSep2012.pdf}}
    \subfigure[]{\includegraphics[width=.5\textwidth]{Digitizer/figs/nhit_e-_50GeV_AugSep2012.pdf}}
    \caption{Distribution de nombre de coups pour des échantillons d'électrons de 20 GeV (a) et 40 GeV (b). Les données sont représentées par des croix noires et la simulation par les histogrammes rouges.}
  \label{fig.nhite-_dist}
\end{figure}
La figure~\ref{fig.nhite-_dist} présente les distributions de nombre de coups pour des échantillons de gerbes électromagnétiques à 20 et 50 $GeV$ pour les données et la simulation. 
\begin{figure}[!ht]
  \centering
  \includegraphics[width=0.5\textwidth]{Digitizer/figs/NHITELECTRON.pdf}
  \caption{(Moyenne du nombre de coups pour des échantillons d'électrons en fonction de l’énergie du faisceau. Les données sont représentées par des croix noires et la simulation par des cercles rouges (FTFP\_BERT\_HP) et des carrés bleus (QGSP\_BERT\_HP). La déviation relative est aussi présentée.}
  \label{fig.nhite-}
\end{figure}
La figure~\ref{fig.nhite-} montre les valeurs moyennes de nombre de coups dans les gerbes électromagnétiques en fonction de l'énergie pour les données et la simulation. Les déviations relatives entre les données et la simulation définit comme $\frac{N_{hit}^{sim}-N_{hit}^{data}}{N_{hit}^{data}}$ sont aussi indiquées sur cette figure. L'accord entre les données expérimentales et les deux listes physiques utilisées pour la simulation est très satisfaisant. Les déviations relatives sont inférieurs à 3\% sur toute la gamme d'énergie. Ces résultats ont tendance à valider l'algorithme de modélisation et sa paramétrisation. 

%%%%%%%%%%%%%%%%%%%%%

\subsubsection{Gerbes hadroniques}
La même procédure de sélection que celle décrite dans le chapitre~\ref{chap.sdhcal} est appliquée sur les données expérimentales et sur la simulation. La figure~\ref{fig.pi-nhit} montre les distributions du nombres total de coups pour des gerbes hadroniques à 20, 30, 60 et 80 $GeV$ pour les données et la simulation. Sur cette figure, la simulation est réalisée avec la liste physique FTFP\_BERT\_HP. L'accord entre données et simulation semble raisonnable à basse énergie. À haute énergie, la simulation sous-estime le nombre total de coups des gerbes hadroniques.
\begin{figure}[!ht]
  \subfigure[]{\includegraphics[width=.5\textwidth]{Digitizer/figs/nhit_pi-_20GeV_AugSep2012.pdf}}
  \subfigure[]{\includegraphics[width=.5\textwidth]{Digitizer/figs/nhit_pi-_40GeV_AugSep2012.pdf}}
  \subfigure[]{\includegraphics[width=.5\textwidth]{Digitizer/figs/nhit_pi-_60GeV_AugSep2012.pdf}}
  \subfigure[]{\includegraphics[width=.5\textwidth]{Digitizer/figs/nhit_pi-_80GeV_AugSep2012.pdf}}
  \caption{Distribution du nombre de coups pour des échantillons de pions de 20 GeV (a), de 40 GeV(b), de 60 GeV (c) et de 80 GeV (d). Les données sont représentées par des croix noires et la simulation (FTFP\_BERT\_HP) par les histogrammes rouges. \label{fig.pi-nhit}}
\end{figure}
Cette impression est confirmée par la figure~\ref{fig.nhit_pi-_ebeam}(a) qui montre le nombre total de coups moyen dans les gerbes hadroniques en fonction de l'énergie du faisceau pour les données et deux liste physiques de simulation. La déviation relative est aussi montré sur cette figure. Un bon accord entre la simulation et les données est trouvé jusque 30 $GeV$. À partir de 40 $GeV$, l'accord se dégrade. La déviation relative entre les données et les deux liste physiques est environ de 15$\%$ à partir de 60$GeV$.
\begin{figure}[!ht]
  \subfigure[]{\includegraphics[width=.5\textwidth]{Digitizer/figs/NHITPIONHP.pdf}}
  \subfigure[]{\includegraphics[width=.5\textwidth]{Digitizer/figs/NHITPROTON.pdf}}
  \caption{Moyenne du nombre de coups pour des gerbes hadroniques en fonction de l'énergie du faisceau. Les déviations relatives sont aussi présentée. (a): Les données sont représentés par des croix noires et la simulation par des cercles rouges (FTFP\_BERT\_HP) et des carrés bleus (QGSP\_BERT\_HP). (b): Les données sont représentés par des croix noires et la simulation par des cercles rouges pour les pions et des triangles verts pour les protons (FTFP\_BERT\_HP).}
  \label{fig.nhit_pi-_ebeam}
\end{figure}
Pour essayer d'expliquer ces résultats, nous avons d'abord pensé à la contamination des données par les protons. La figure~\ref{fig.nhit_pi-_ebeam}(b) montre le nombre total de coups moyen pour les données, pour une simulation de pion et de proton avec la liste physique FTFP\_BERT\_HP. Au dessus de 40 $GeV$, le nombre de coups total est légèrement plus élevé pour les gerbes hadroniques initiées par des protons que par des pions. Cependant le nombre de coups pour une simulation de proton est toujours significativement plus faible que dans les données à haute énergie. La déviation relative est supérieure à 10$\%$ au dessus de 60 $GeV$. D'autres listes physiques préparées par GEANT4 sont aussi testées. La figure~\ref{fig.nhit_pi-_ebeam_model}(a) montre le nombre de coups moyen en fonction de l'énergie pour les listes physiques QGSP\_BERT et FTFP\_BERT avec et sans l'option de haute précision pour les neutrons. Le nombre de coups est très légèrement supérieur lorsque cette option n'est pas utilisée. La figure~\ref{fig.nhit_pi-_ebeam_model}(b) présente aussi le nombre de coups moyen en fonction de l'énergie pour les listes physiques FTF\_BIC, QGSP\_FTFP\_BERT et QGSP\_BIC. La liste FTF\_BIC est en meilleurs accord avec les données expérimentales, particulièrement à haute énergie. Au delà de 60 $GeV$, le nombre de coups reste significativement sous-estimer.
\begin{figure}[!ht]
  \subfigure[]{\includegraphics[width=.5\textwidth]{Digitizer/figs/NHITPION_HP.pdf}}
  \subfigure[]{\includegraphics[width=.5\textwidth]{Digitizer/figs/NHITPION_MODEL.pdf}}
  \caption{Moyenne du nombre de coups pour des gerbes hadroniques en fonction de l'énergie du faisceau. Les déviations relatives sont aussi présentée. (a): effet de l'utilisation du modèle de haute précision pour les neutrons pour les listes physiques QGSP\_BERT et FTFP\_BERT. (b): comparaison entre données et les résultats avec différentes listes physiques.}
  \label{fig.nhit_pi-_ebeam_model}
\end{figure}

La figure~\ref{fig.pi-nhit_thr} montre le nombre de coups moyen pour chaque seuil, pour les données et la simulation. Le nombre de coups pour les seuils 1 et 2 respectent la même tendance que pour le nombre de coups total: l'accord entre les données et la simulation est raisonnable à basse énergie et se dégrade sensiblement au dessus de 40 $GeV$. Le nombre de coups pour le troisième seuil est la variable la plus sensible au fluctuation de température et de pression. Ces fluctuations sont équivalent à des  variations de tension dans le gaz. Ceci permet d'expliquer le comportement de cette variable pour les données. Il est ainsi délicat de tirer des conclusions avec cette variable. Cependant, on peut constater que la liste physique QGSP\_BERT\_HP produit légèrement plus de coups seuil 3 que la liste FTFP\_BERT\_HP. Ceci est dus à la gamme de validité du modèle de Bertini qui est plus étendue pour la liste QGSP\_BERT\_HP que pour FTFP\_BERT\_HP. Ce modèle à tendance à créer beaucoup de particules secondaires et donc d'augmenter les coups seuil 3.
\begin{figure}[!ht]
  \includegraphics[width=.32\textwidth]{Digitizer/figs/NHIT1PIONHP.pdf}
  \includegraphics[width=.32\textwidth]{Digitizer/figs/NHIT2PIONHP.pdf}
  \includegraphics[width=.32\textwidth]{Digitizer/figs/NHIT3PIONHP.pdf}
  \caption{Nombre moyen de coups pour chaque seuil en fonction de l'énergie du faisceau. Les données sont représentés par des croix noires et la simulation par des cercles rouges (FTFP\_BERT\_HP) et des carrés bleus (QGSP\_BERT\_HP).\label{fig.pi-nhit_thr}}
\end{figure}

%%%%%%%%%%%%%%%%%%%%%%%%%%%%%%%%%%%%%%%%%%%%%%%

\section{Conclusion}
Les différentes étapes de la simulation du prototype et de la modélisation de la réponse des GRPCs ont été détaillées. La paramétrisation de l'algorithme SimDigital a été réalisée grâce aux études menées sur la réponse des muons dans le détecteur. Les gerbes électromagnétiques ont aussi été utilisées pour cette paramétrisation. Les comparaisons entre données expérimentales et la simulation de gerbes électromagnétiques dans le SDHCAL montrent un accord très satisfaisant. Cela permet de valider la procédure de simulation du prototype et l'algorithme de modélisation de sa réponse aux particules chargées. Cependant des désaccords sont observés entre les données et la simulation des gerbes hadroniques. Des études similaires ont été menées par la Collaboration ATLAS~\cite{Abat} et CALICE~\cite{geant4-ahcal} avec des calorimètre utilisant une autre technologie. Ces études ont montré que les listes physiques FTFP\_BERT\_HP et QGSP\_BERT\_HP simulent correctement la réponse de ces détecteurs aux passages de gerbes hadroniques. Cependant ces deux calorimètres utilise des scintillateurs comme milieu actif, qui ont une lecture analogique. Le signal de ces scintillateurs est proportionnel à l'énergie déposée par les particules chargées. De plus, la segmentation transverse de ces deux calorimètre est beaucoup moins fine que pour le SDHCAL. Ceci pourrait expliquer pourquoi la réponse simulée dans le SDHCAL est plus faible que dans les données alors que celle-ci est en bon accord pour les détecteurs ATLAS-TileCal et CALICE-AHCAL.
